\documentclass{life-fr}

\begin{document}

\title{La Vie est un Jeu}
\subtitle{Documentation Technique}

\member{Lepage Barbara}{lepage.barbara@gmail.com}
\member{Caradec Guillaume}{guillaume.caradec@gmail.com }
\member{Corsin Simon}{simoncorsin@gmail.com }
\member{Glorieux François}{fra.glorieux@gmail.com}
\member{Klarman Nicolas}{nickoas@gmail.com}
\member{Lassagne David}{david.lassagne@gmail.com}
\member{Louvigny Guillaume}{guillaume@louvigny.fr}
\member{Le-Cor Wilfried}{wilfried.lecor@gmail.com}
\member{Lenormand Frank}{lenormf@gmail.com}
\member{El-Outmani Youssef}{youssef.eloutmani@gmail.com}

\summary
{
  Résumé CONFIDENTIEL des composants techniques du projet destiné aux développeurs
}

\maketitle

%% --------------------------------------------------------------------- %%

\chapter*{CONFIDENTIEL}
{
  Ce document est CONFIDENTIEL et interne au projet «~La Vie Est Un Jeu~».\\
  \\
  Sauf autorisation exceptionnelle, les seuls personnes autorisées à le lire
  sont les membres officiels actuels du projet ainsi que l'équipe des
  professeurs du laboratoire des EIP (Epitech Innovative Projects) à Epitech.\\
  \\
  Les seules personnes habilitées à donner une autorisation exceptionelle
  ou à partager ce fichier sont les chefs de projets du projet «~La Vie
  Est Un Jeu~».
}

%% --------------------------------------------------------------------- %%

\authorspage

%% --------------------------------------------------------------------- %%

\chapter*{Résumé}
{
  « La Vie Est Un Jeu » propose des applications mobiles, tablettes et
  web pour transformer votre quotidien en un jeu dans lequel vous vous
  défiez avec des objectifs et des badges en récompense.\\
  Notre Cahier Des Charges donne toutes les informations supplémentaires.\\
  \\
  Les développeurs sont réparties en équipes au sein du groupe.\\
  Nous utilisons un certain nombre d'outils d'organisation regroupés sur
  \href{http://life.paysdu42.fr/}{un portail}, notamment des agendas, des
  réunions hebdomadaires, des dépôts Gitub et des ``issues'' Github pour gérer
  les tâches.\\
  \\
  Les équipes de développeurs correspondent aux différents projets constituant
  notre projet :\\
  \begin{description}
    \item [Web service respectant une API] :\\
      Ce web service est utilisé par les clients mobiles, tablettes, web et
      externes pour manipuler la base de données. Il est développé avec
      \textbf{Javascript}, \textbf{Node.JS} et \textbf{MongoDB}.
    \item [Applications mobiles et tablettes] :\\
      Ces applications sont disponibles sur Android, iOs et Windows Phone
      grâce à la technologie \textbf{Mono}.
    \item [Site Web] :\\
      Le site web est utilisable sur mobile, tablette ou ``desktop''. Il
      est développé avec \textbf{Ocsigen} en \textbf{OCaml}.
    \item [Documentation] :\\
      De nombreuses documentations parsement notre projet, comme les
      documentations dites ``globales'' (demandées par le lab EIP) en
      \textbf{LaTeX}, ``internes'' (correspondant aux dépôts) en
      \textbf{Markdown} ou ``dans le code'' utilisant \textbf{Doxygen}.
  \end{description}

  \vspace{10pt}
  Cette documentation ne suffit pas à elle-seule, les développeurs
  doivent prendre le temps de lire toutes les documentations liées.
}

%% --------------------------------------------------------------------- %%

\chapter*{Informations du document}

\rowcolors{1}{white}{lightgray}
\begin{tabular}{ | m{5cm} | m{10cm} | }
  \hline
  \textbf{Type du document} & Documentation Technique\\ % to change
  \hline
  \textbf{Titre complet du document} & Documentation Technique «~La Vie Est Un Jeu~» pour nouveaux développeurs\\ % to change
  \hline
  \textbf{Mots clés} & «~technique~», «~développeur~», «~API~», «~lavieestunjeu~»\\ % to change
  \hline
  \textbf{Nombre de pages} & \ref{TotPages} \\
  \hline
  \textbf{Nom du groupe} & La Vie Est Un Jeu\\
  \hline
  \textbf{Responsable} & Chef de groupe : Barbara Lepage\\
  \hline
  \textbf{Membres du groupe} & Voir page de garde\\
  \hline
  \textbf{Auteurs} & Membres du groupe, voir page de garde\\
  \hline
  \textbf{Contact} & lavieestunjeu@googlegroups.com\\
  \hline
  \textbf{Révision actuelle} & 1.0\\ % to change
  \hline
  \textbf{Site vitrine} & \url{http://eip.epitech.eu/2014/lavieestunjeu/}\\
  \hline
  \textbf{Site officiel} & Non disponible\\
  \hline
\end{tabular}

%% --------------------------------------------------------------------- %%

\chapter*{Table des révisions}

\revision{0.1}{Lepage Barbara}{Mise en page}{15/03/2013}
\revision{0.2}{Lepage Barbara}{Introduction}{15/03/2013}
\revision{0.3}{Lepage Barbara}{Outils et pré-requis}{15/03/2013}
\revision{0.4}{Lepage Barbara}{API}{15/03/2013}
\revision{0.5}{Lepage Barbara}{Applications mobiles et tablettes}{20/03/2013}
\revision{0.6}{Lepage Barbara}{Site web}{20/03/2013}
\revision{0.7}{Lepage Barbara}{Documentation}{20/03/2013}
\revision{0.8}{Lepage Barbara}{Conclusion}{20/03/2013}
\revision{0.9}{Lepage Barbara}{Résumé du document}{20/03/2013}
\revision{1.0}{Lepage Barbara}{Première version}{20/03/2013}
\listofrevisions

%% --------------------------------------------------------------------- %%
\newpage

\tableofcontents

%% --------------------------------------------------------------------- %%

\chapter{Introduction}

Ce document s'adresse aux nouveaux développeurs rejoignant l'équipe de
développement du projet «~La Vie Est Un Jeu~».\\
Nous présumons que les lecteurs de ce document ont déjà pris connaissance
du document ``Cahier Des Charges'' et ont des bases en programmation.\\
\\
Notre application se décline en plusieurs composants :
\begin{itemize}
  \item Une API publique développeur
  \item Des applications mobiles et tablettes
  \item Un site web
\end{itemize}
Il n'est pas nécéssaire pour vous de lire l'intégralité du document
si vous ne participez qu'à un des composants, mais nous le recommendons
afin d'être en phase avec le reste de l'équipe et avoir une vision globale
du projet.\\
\\

%% --------------------------------------------------------------------- %%

\chapter{Outils et pré-requis}

\section{Portail}

Comme nous sommes de nombreux développeurs, nous nécessitons une organisation
exemplaire pour mener à bien notre projet. Pour cela, nous utisons un portail
qui regroupe tous nos outils.\\
\\
Le portail est disponible à l'URL suivante :
\url{http://life.paysdu42.fr/}\\
Il est nécessaire pour tous les développeurs de l'équipe de s'y rendre tous
les jours de travail sur le projet.\\
\\
Les sources du portail sont disponibles à l'URL suivante :
\url{https://github.com/LaVieEstUnJeu/Internal-Tools}\\
Plus d'informations sur les dépôts dans la partie ``Sources des projets''.

\subsection{Organisation}

Les outils que nous utilisons sont tous sur le portail, mais aller plus
loin dans leur utilisation permet plus d'efficacité. Nous établissons
un bilan des tâches en cours, en créons de nouvelles ou étendons au
besoin les dates limites.

\subsubsection{Agenda}

Notre agenda est hebergé par \textbf{Google Calendar}. Il est fortement
recommendé de l'ajouter dans votre propre système d'agenda. Le format
ICAL que propose Google est compatible avec la plupart des gestionnaires
d'agenda du marché.\\
\\
Nous utilisons également un agenda global qui correspond à notre
diagramme de GANTT.

\subsubsection{Réunions régulières}

Nous organisons des réunions régulières, notamment une réunion technique
par semaine, une réunion brainstorm sur le contenu de l'application par semaine et
des réunions mensuelles ou bi-mensuelles organisées par le lab EIP
(communication, suivis, soutenances techniques, soutenance communication).\\
\\
Pendant la \textbf{réunion technique hebdomadaire}, nous abordons nos
avancements respectifs sur le projet puis nous nous séparons par
équipes pour plus d'efficacité.

\subsubsection{Communication}

Notre \textbf{mailing-list} \href{mailto:lavieestunjeu@googlegroups.com}{lavieestunjeu@googlegroups.com}
est notre principal outil de communication (en français), mais tout
ce qui concerne le côté technique de notre projet DOIT être sur les
``issues'' GitHub (voir ci-dessous) et rédigé en anglais.\\
\\
Nous communiquons également régulièrement sur GTalk, IRC, par téléphone
et Google Hangout.

\subsubsection{Serveur de développement}

Comme certaines technologies que nous utilisons, notamment Ocsigen, ne
sont pas portables et sont très difficiles à installer, nous utilisons
un serveur de développement commun sur lequel sont installés les
logiciels et librairies nécessaires aux développeurs.

\section{Sources des projets}

\subsection{Git}

Tous nos projets sont versionnés en utilisant le gestionnaire de version
\textbf{Git}.\\
\\
De nombreuses références existent sur l'Internet pour ces notions, notamment
\href{http://fr.wikipedia.org/wiki/Gestion_de_versions}{la page Wikipédia
gestion de version}
et \href{http://git-scm.com/}{le site officiel Git}.

\subsection{GitHub}

L'hébergeur de nos dépôts Git s'appelle \textbf{GitHub}. Ce service
propose des dépôts publics gratuits et des dépôts privés payants,
ainsi que de nombreuses autres fonctionnalités.\\
\\
Les fonctionnalités principales que nous utilisons sont les suivantes :
\begin{description}
  \item [Organizations] : \\
    Les ``organizations'' GitHub permettent de regrouper des dépôts
    sur une même page et de gérer des équipes.\\
    Notre ``organization'' : \url{https://github.com/LaVieEstUnJeu}.
  \item [Issues] : \\
    Les ``issues'' peuvent être comparés à des ``tickets''. Ils permettent
    de répertorier et discuter les fonctionnalités en cours de développement,
    les bogues, de poser des questions, de proposer des nouvelles fonctionnalités
    ou de gérer les ``pull-request'' (demandes extérieures d'ajout de
    modifications sur le dépôt principal).\\
\end{description}

\subsection{Accès aux dépôts}

La majorité de nos dépôts est privée. Pour y avoir accès, il faut être membre
de l'équipe de développement du projet en question.\\
\\
Si vous n'êtes pas encore membre des équipes vous permettant d'avoir accès
aux dépôts, faites une demande au plus vite.\\
En attendant, afin de permettre aux nouveaux développeurs n'ayant pas encore
accès à nos dépôts (et au lab EIP) de parcourir le contenu de nos dépôts,
nous avons créer un compte utilisateur \textbf{lavieestunjeu-watcher}. Le
mot de passe de ce compte est ``2h\&1H52Z''.\\
\\
Les équipes actuelles sont :
\begin{description}
  \item [Watchers] : \\
    Personnes extérieures au projet ayant une autorisation exceptionnelle
    pour \textbf{voir} certains dépôts (exemple : jury dans le cadre des
    EIP, conseillers officiels, lab EIP).
  \item [Web Service API] : \\
    Développeurs du Web Service respectant l'API.
  \item [Mobile] : \\
    Développeurs des applications mobiles et tablettes.
  \item [Website] : \\
    Développeurs du client web.
\end{description}

\subsection{Règles d'utilisation des dépôts}

\subsubsection{Règles générales}

\begin{itemize}
  \item Le contenu des dépôts privés est confidentiel et ne doit en
    aucun cas être diffusé.
  \item Les contenu des dépôts publics sont protégés par leurs licences
    open-source respectives, présentes directement dans les dépôts.
  \item Afin de faciliter l'insertion de nouveaux membres de nationalités
    différentes, nous utilisons l'\textbf{anglais} sur ces dépôts.\\
    Aucun mot français ou d'une autre langue n'est toleré, que ce soit
    dans les sources, dans les ``issues'' ou dans les messages de ``commit''.
  \item Chaque dépôt et chaque sous-projets dans les dépôts doit contenir
    un fichier \textbf{README}. Voir section ``Documentation'' pour
    plus d'informations.
\end{itemize}

\subsubsection{``Commit''}

\begin{itemize}
  \item Les commits répondant à une ``issue'' doivent la citer avec un ``\#''
  \item Les messages de commit doivent toujours être explicites
  \item Il est recommendé de segmenter les messages de commit avec un
    résumé de moins de 50 lettres suivi de deux retours à la ligne
    et d'une description complète (souvent une liste)
  \item Un commit doit contenir une et une seule fonctionnalité
  \item Dans le cas où les dépôts contiennent plusieurs projets,
    il est impératif de mettre un ``tag'' au début du message de commit\\
    Exemple : ``[Portal] Add labels and sub-labels in the menu''
  \item Les ``diff'' de commit ne doivent jamais contenir de lignes
    hors-sujet.
\end{itemize}

\subsubsection{``Issues''}

\begin{itemize}
  \item Il doit toujours y avoir une ``issue'' ouverte pour chaque tâche en cours
  \item Les issues contiennent des ``tags'' :
    \begin{itemize}
      \item task
      \item bug
      \item bonus
      \item question
    \end{itemize}
    \item Nous n'utilisons pas d'issues dans les dépôts publics
\end{itemize}


%% --------------------------------------------------------------------- %%

\chapter{API Publique développeur}

\section{L'API}

L'interraction entre les données que nous manipulons et nos services est
strictement définie par une API.

\subsection{Spécifications}

\subsubsection{Architecture REST}

Cette API utilise le style d'architecture standardisé \textbf{REST}
(``REpresentational State Transfer'').\\
\\
De nombreuses références existent sur l'Internet pour ces notions, notamment
\href{http://fr.wikipedia.org/wiki/Representational_State_Transfer}{la page Wikipédia}.

\subsubsection{Formattage JSON}

Les données publiées par l'API sont formattées en utilisant le format
de données standardisé \textbf{JSON} (``JavaScript Object Notation'').\\
\\
De nombreuses références existent sur l'Internet pour ces notions, notamment
\href{http://www.json.org/}{le site officiel}.

\subsection{Documentation de l'API}

Notre API évolue régulièrement, contrairement à cette documentation, c'est
pourquoi elle n'est pas directement reportée sur celui-ci.\\
\\
Elle est disponible sur un Google Document : \url{http://goo.gl/uxMoJ}.\\
Lorsque notre service sera publié, l'API sera disponible sur notre site
web.

\section{Le Web Service}

Nous proposons à nos utilisateurs un web service respectant l'API leur
permettant de manipuler les données de nos services. Nos clients web et
mobiles sont également basés sur ce Web Service pour la manipulation
de données.\\
\\
Pour plus d'informations sur ce qu'est un Web Service, de nombreuses
références existent sur l'Internet pour ces notions, notamment
\href{http://fr.wikipedia.org/wiki/Service_Web}{la page Wikipédia}.

\subsection{Sources}

\subsubsection{Languages de programmation}

Notre Web Service est implémenté en \textbf{Javascript}
avec le framework \textbf{Node.js}.\\
\\
Le choix de cette technologie est détaillé dans le cahier des charges.\\
\\
De nombreuses références existent sur l'Internet pour ces notions, notamment
\href{http://fr.wikipedia.org/wiki/JavaScript}{la page Wikipédia Javascript}
et \href{http://nodejs.org/}{le site officiel Node.js}.

\subsubsection{Dépôts}

Le dépôt GitHub du Web Service : \url{https://github.com/LaVieEstUnJeu/API}\\
Il est primordial de prendre conaissance des informations fournies dans le
``README''.\\
Plus d'informations sur les dépôts dans la partie ``Sources des projets''.

\subsection{Base de données}

Nous utilisons une base de donnée dite ``non relationelle'' appelée MongoDB.\\
\\
De nombreuses références existent sur l'Internet pour ces notions, notamment
\href{http://fr.wikipedia.org/wiki/NoSQL}{la page Wikipédia NOSQL} et
\href{http://www.mongodb.org/}{le site officiel MongoDB}.\\
\\
La base de données évolue au fil de l'avancement du projet et nous mettons
à jour l'API régulièrement, tout au long de la phase de développement.\\
\\
Le schéma théorique de la base de donnée est disponible dans le document
architecture du projet : \url{https://github.com/LaVieEstUnJeu/Doc/tree/master/doc/AA}\\
\\
La version actuelle - mais pas finale - de la base de donnée MongoDB est
disponible sur le dépôt :
\url{https://github.com/LaVieEstUnJeu/API/blob/master/public/models.js}.

\section{Utilisation du Web Service}

Notre Web Service est utilisé par nos clients mobiles et webs et
potentiellement par d'autres développeurs par le futur. Pour cette raison,
nous avons décidé de publier le code source de l'utilisation de l'API
puis par la suite de proposer d'autres exemples afin que les utilisateurs
soient guidés au mieux.\\
\\
Le dépôt \textbf{public} de l'utilisation de l'API :
\url{https://github.com/LaVieEstUnJeu/Public-API/}\\
Il est primordial de prendre conaissance des informations fournies dans le
``README'' et tous les autres ``README'' disponibles pour chaques exemples
fournis.\\
Plus d'informations sur les dépôts dans la partie ``Sources des projets''.\\
\\
Le code source de ces exemples n'est disponible QUE sur ce dépôt afin de
ne pas faire de dupplication. Pour cette raison, les dépôts utilisant ce
code doivent y faire appel lors de leur installation.

%% --------------------------------------------------------------------- %%

\chapter{Applications mobiles et tablettes}

\section{Supports et technologie}

Nos applications sont disponibles sur mobiles et tablettes, sur plusieurs
systèmes d'exploitation.\\
\\
Afin de minimiser la quantité de code produite et ainsi limiter la
quantité de bogues potentiels, nous utilisons la technologie \textbf{Mono}
en C\#.\\
En particulier, nous utilisons :
\begin{itemize}
  \item Code natif C\# pour \textbf{Windows Phone 7}
  \item Mono.android pour \textbf{Android}
  \item MonoTouch pour \textbf{iOs}
\end{itemize}

Le choix de cette technologie est détaillé dans le cahier des charges.\\
\\
De nombreuses références existent sur l'Internet pour ces notions, notamment
\href{http://msdn.microsoft.com/en-us/vstudio/hh341490.aspx}{la documentation
officielle C\#} et \href{http://www.mono-project.com/}{le site officiel Mono}.

\section{Source}

\subsection{Dépôts}

Le dépôt GitHub des applications :
\url{https://github.com/LaVieEstUnJeu/Applications}\\
Plus d'informations sur les dépôts dans la partie ``Sources des projets''.

\subsection{Tests unitaires de l'API}

Le dépôt Applications propose également des tests unitaires pour l'implémentation
côté client de l'API.

\section{Design}

Le design des vues ainsi que les déscriptif des fonctionnalités de celles-ci
sont définis dans ce document CONFIDENTIEL :
\url{http://goo.gl/oY9se}

%% --------------------------------------------------------------------- %%

\chapter{Site Web}

\section{Technologie}

Le site web est développé avec \textbf{Ocsigen}, serveur et framework web
en \textbf{OCaml}.\\
\\
Le choix de cette technologie est détaillé dans le cahier des charges.\\
\\
De nombreuses références existent sur l'Internet pour ces notions, notamment
\href{http://ocaml.org/}{le site officiel OCaml} et
\href{http://ocsigen.org/}{le site officiel Ocsigen}.\\
\\
Le code client et le code serveur sont développés
avec la même technologie. Les vues sont ``responsives'', c'est à dire
que l'affichage est automatiquement adapté pour une visualisation sur
un environnement dit ``desktop'', ``mobile'' ou ``tablette''.

\section{Dépôts}

Le dépôt GitHub du site web :
\url{https://github.com/LaVieEstUnJeu/Website}\\
Il est primordial de prendre conaissance des informations fournies dans le
``README''.\\
Plus d'informations sur les dépôts dans la partie ``Sources des projets''.

\section{URLs}

Les URL sont définies sur le ``README'' :
\url{https://github.com/LaVieEstUnJeu/Website#urls}

\section{Design des vues}

Le design des vues est défini dans cet album photo CONFIDENTIEL :
\href{http://photos.db0.fr/?album=GLifeDesign?authkey=Gv1sRgCLmfvp7DmMK_hAE}{album}.

\section{Fonctionnalités}

Le descriptif des fonctionnalités sur les vues est défini dans le cahier
des charges.

%% --------------------------------------------------------------------- %%

\chapter{Documentation}

Ce chapitre s'adresse aux rédacteurs techniques du projet.\\
La section ``Dans le code'' s'adresse à tous les développeurs.

\section{Documentation globales du projet}

\subsection{Sources}

Le dépôt GitHub des documentations globales du projet :
\url{https://github.com/LaVieEstUnJeu/Applications}\\
Il est primordial de prendre conaissance des informations fournies dans le
``README''.\\
Plus d'informations sur les dépôts dans la partie ``Sources des projets''.

\subsection{Étapes et technologies}

Nous documentons nos projets selon les désirs du lab EIP et ajoutons
également d'autres documentations nous semblant utiles.

\subsubsection{Rédaction}

Au besoin, nous rédigons le contenu de nos documents en utilisant un
outil collaboratif (Google Documents/Drive) nous permettant de
travailler à plusieurs et à distance.\\
\\
Si une seule personne travaille sur le document, il n'est pas
nécessaire d'utiliser cet outil et nous utilisons directement les
outils de mise en page.

\subsubsection{Mise en page}

Nous mettons en page nos documents en utilisant \textbf{LaTeX},
un language de balisage (communément appelé ``markup language'' en anglais).\\
\\
De nombreuses références existent sur l'Internet pour ces notions, notamment
\href{http://fr.wikipedia.org/wiki/Langage_de_balisage}{la page wikipédia} et
\href{http://www.latex-project.org/}{le site officiel LaTeX}.

\section{Documentations directement liées aux dépôts}

Chaque dépôt et chaque sous-projets dans les dépôts doit contenir
une documentation dans un fichier ``README''.

\subsection{Contenu}

Ce fichier doit contenir :
\begin{itemize}
  \item Le contenu, le but du dépôt/projet
  \item Les dépendances du projet
  \item Une manière simple et rapide d'installer, compiler, lancer, tester
  \item L'arborescence et l'emplacement des fichiers principaux
  \item Une explication destinées aux personnes qui rejoignent le projet
  \item Le ``coding-style'' (ou convention de codage) du projet
  \item Une F.A.Q
  \item Le copyright (pour les dépôts open-source)
\end{itemize}

\subsection{Sources}

Ces documentations doivent être rédigé en utilisant le language
de balisage \textbf{Markdown} (recommendé) ou un autre
language supporté par GitHub.\\
\\
La liste des languages supportés par GitHub et les liens vers
leurs documentations sont disponibles sur
\href{https://github.com/github/markup#markups}{la documentation GitHub}.

\section{Dans le code}

Il est primordial de commenter son code et de le documenter au mieux.
Pour cela, il faut garder en tête que d'autres personnes auront besoin
de repasser derrière pour le lire, le comprendre et le modifier et qu'ils
doivent pouvoir le faire sans difficulté et sans l'aide du developpeurs
original.\\
\\
Nous recommendons de manière générale d'utiliser la syntaxe \textbf{Doxygen}
pour générer automatiquement une documentation, mais chaque projet
a ses propres conventions de codage définies dans les ``README''.\\
\\
De nombreuses références existent sur l'Internet pour ces notions, notamment
\href{http://fr.wikipedia.org/wiki/R%C3%A8gles_de_codage}{la page Wikipédia
conventions de codage} et \href{http://www.doxygen.org/}{le site officiel Doxygen}.

%% --------------------------------------------------------------------- %%

\chapter{Aller plus loin}

Ce document est un guide simple permettant aux nouveaux développeurs
de rejoindre le projet sans difficulté, mais il ne constitue pas à lui
seul une documentation complète.\\
\\
Il est donc essentiel qu'en tant que développeurs, vous preniez le temps
de vous documenter sur les projets sur lesquels vous travaillez en utilisant
les liens proposés dans ce document et en faisant vos recherches personnelles.\\
\\
Il est également primordial pour la bonne cohésion du groupe de communiquer
au sein des équipes. Pour cette, la participation aux réunions est obligatoire
et permet de poser des questions ou d'y répondre.\\
\\
Pour finir, il est important que vous preniez le temps, comme l'ont fait
et le feront toute l'équipe, de rédiger les documentations de vos projets
respectifs et que vous les teniez à jour.

\end{document}
