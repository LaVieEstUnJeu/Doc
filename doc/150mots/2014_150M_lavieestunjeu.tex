\documentclass{life-fr}

\begin{document}

\title{La vie est un jeu}
\subtitle{50 Mots}
\member{Lepage Barbara}{db0company@gmail.com}
\member{Caradec Guillaume}{guillaume.caradec@gmail.com }
\member{El-Outmani Youssef}{youssef.eloutmani@gmail.com}
\member{Glorieux François}{fra.glorieux@gmail.com}
\member{Klarman Nicolas}{nickoas@gmail.com}
\member{Lassagne David}{david.lassagne@gmail.com}
\member{Le-Cor Wilfried}{wilfried.lecor@gmail.com}
\member{Lenormand Frank}{lenormf@gmail.com}
\member{Louvigny Guillaume}{guillaume@louvigny.fr}

\summary
{
  Les 150 mots sont là pour décrire notre projet succinctement
  afin de nous apprendre à le synthétiser pour l'expliquer à
  une personne qui n'est pas forcément dans le domaine de
  l'informatique et plus généralement le domaine de notre EIP.
}

\maketitle


\chapter{La vie est un jeu en 150 mots}

Envie de pimenter ta vie ? Relève le défi ! Avec gLife, tu mets ta vie en jeu et deviens un héro exceptionnel.
Tes tâches quotidiennes deviennent tes défis, tes amis deviennent les témoins de tes actes héroïques.

Comment ça fonctionne ? Une fois inscrit, gLife te propose de réaliser les défis de ton choix,
dans la période de temps que tu désires. Une fois ces défis réalisés,
« Achievement Unlocked !». Partage-les avec ton réseau pour qu’ils commentent et attestent
de la bonne réussite de tes exploits, t’offrant pour la même occasion un badge et des points
à la hauteur de ta prouesse. Compare tes réussites avec tes amis,
et défis-les d’en faire de nouveaux, seuls ou en groupes !

Tu n’es qu’à une poignée de clics de commencer le grand jeu de ta vie.
Alors, rejoins la communauté de gLife et deviens un gLifer !

\end{document}
