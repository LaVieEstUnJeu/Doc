\documentclass{life-fr}
\usepackage{eurosans}
\usepackage{totpages}

\begin{document}

\title{Life is A Game}
\subtitle{Technical Documentation}

\member{Lepage Barbara}{lepage.barbara@gmail.com}
\member{Caradec Guillaume}{guillaume.caradec@gmail.com }
\member{Corsin Simon}{simoncorsin@gmail.com }
\member{Glorieux François}{fra.glorieux@gmail.com}
\member{Klarman Nicolas}{nickoas@gmail.com}
\member{Lassagne David}{david.lassagne@gmail.com}
\member{Louvigny Guillaume}{guillaume@louvigny.fr}
\member{Le-Cor Wilfried}{wilfried.lecor@gmail.com}
\member{Lenormand Frank}{lenormf@gmail.com}


\summary
{
  CONFIDENTIAL summary of the technical components of the project for developers
}

\maketitle

%% --------------------------------------------------------------------- %%

\chapter*{CONFIDENTIAL}
{
  This document is CONFIDENTIAL and internal for the project «~Life Is A Game~».\\
  \\
  Without a special agreement, the only persons authorized to read
  this document are the official members of the current project and the team
  of professors of the EIP Laboratory (Epitech Innovative Projects) at Epitech.\\
  \\
  The only persons authorized to give exceptional permission
  or share this file are project managers of the project «~Life Is A Game~»
  
}

%% --------------------------------------------------------------------- %%

\authorspageen

%% --------------------------------------------------------------------- %%

\chapter*{Abstract}
{
  "Life Is A Game" offers mobile and tablets applications, and
  websites to transform your life into a game in which you
  challenge yourself with goals and reward badges.\\
  The "Specifications" document provides all additional information.\\
  \\
  Developers are divided into teams within the group.\\
  We use a number of tools for organizing grouped on
  \href{http://life.paysdu42.fr/}{a portal}, including agendas
  weekly meetings, repositoryions and Gitub ``from'' Github to manage
  tasks.\\
  \\
  Development teams correspond to different projects constituting
  our project: \\
  \begin{description}
    \item [respecting a Web service API]: \\
      This web service is used by mobile clients, tablets, and web
      External to manipulate the database. It is developed with
      \textbf{Javascript} \textbf{Node.JS} and \textbf{MongoDB}x.
    \item [Mobile and Tablets]: \\
      These applications are available on Android, iOS and Windows Phone
      through technology \textbf{Mono}.
    \item [Website]: \\
      The website can be used on mobile, tablet or ``desktop''. It
      developed with \textbf{Ocsigen} in \textbf{OCaml}.
    \item [Documents]: \\
      Many dot our documentation project, such as
      documentation called ``global'' (requested by the lab EIP) in
      \textbf{LaTeX} ``internal'' (corresponding to repositories) in
      \textbf{Markdown} or ``code'' using the \textbf{Doxygen}.
  \end{description}

  \vspace{10pt}
  This documentation is not sufficient to itself alone, developers
  should take the time to read all the documentation related.
}

%% ------------------------------------------------ ---------------------%%

\chapter*{Document Information}

\rowcolors{1}{white}{lightgray}
\begin{tabular}{ | m{5cm} | m{10cm} | }
  \hline
  \textbf{Document Type} & Technical Documentation \\ % to exchange
  \hline
  \textbf{Full title Document} & Technical Documentation «~Life Is A Game~» for new developers \\% to exchange
  \hline
  \textbf{Keywords} & «~Technical~», «~Developer~», «~API~» «~lavieestunjeu~» \\% to exchange
  \hline
  \textbf{Number of pages} & \ref{TotPages} \\
  \hline
  \textbf{Manager} & Group Leader: Barbara Lepage \\
  \hline
  \textbf{Group Members} & See cover page \\
  \hline
  \textbf{Authors} & Group members, see cover Page \\
  \hline
  \textbf{Contact} & lavieestunjeu@googlegroups.com \\
  \hline
  \textbf{Current revision} & 1.0 \\% to exchange
  \hline
  \textbf{Vitrine website} & \url{http://eip.epitech.eu/2014/lavieestunjeu/} \\
  \hline
  \textbf{Official Website} & Not available \\
  \hline
\end{tabular}

%% ------------------------------------------------ ---------------------%%

\chapter *{Revisions table}

\revision{0.1}{Barbara Lepage}{Layout}{15/03/2013}
\revision{0.2}{Barbara Lepage}{Introduction}{3/15/2013}
\revision{0.3}{Barbara Lepage}{Tools and Requirements}{15/03/2013}
\revision{0.4}{Barbara Lepage}{API}{15/03/2013}
\revision{0.5}{Barbara Lepage}{Mobile and tablet}{20/03/2013}
\revision{0.6}{Barbara Lepage}{Website}{20/03/2013}
\revision{0.7}{Barbara Lepage}{Documentation}{20/03/2013}
\revision{0.8}{Barbara Lepage}{Conclusion}{3/20/2013}
\revision{0.9}{Barbara Lepage}{Summary document}{3/20/2013}
\revision{1.0}{Barbara Lepage}{First version}{3/20/2013}
\listofrevisions

%% ------------------------------------------------ ---------------------%%
\newpage

\tableofcontents

%% ------------------------------------------------ ---------------------%%

\chapter{Introduction}

This document is intended for new developers joining the team
development of the project «~Life Is A Game~».\\
We assume that readers of this document have already read
The document ``Specifications'' and programming bases.\\
\\
Our application is divided into several components:
\begin{itemize}
  \item A public developer API
  \item Mobile applications and tablets
  \item A website
\end{itemize}
It is not necessary for you to read the entire document
if you do not participate as a component, but we recommend
to be in line with the rest of the team and have a global vision
the project.\\
\\

%% ------------------------------------------------ ---------------------%%

\chapter{Tools and prerequisites}

\section{Portal}

As we are many developers, we require an organization
copy to complete our project. To do this, we use portal
which includes all of our tools.\\
\\
The portal is available at the following URL:
\url{http://life.paysdu42.fr/} \\
It is necessary for all developers on the team to get to all
days of work on the project.\\
\\
The sources of the portal is available at the following URL:
\url{https://github.com/LaVieEstUnJeu/Internal-Tools} \\
More information about repositories in the ``Sources'' projects.

\subsection{Organization}

The tools we use are all on the portal, but go
far in their use allows more efficiency. We set
an overview of current tasks in creating new or expanding in
need deadlines.

\subsubsection{Calendar}

Our agenda is hosted by \textbf{Google Calendar}. It is strongly
recommended to add in your own calendar. UNIT
ICAL that Google offers is compatible with most managers
calendar market.\\
\\
We also use a global agenda which corresponds to our
GANTT.

\subsubsection{Regular Meetings}

We organize regular meetings, including a technical meeting
a week, a meeting brainstorm inside the application and per week
meetings monthly or bi-monthly lab organized by the EIP
(Communication, monitoring, technical defenses, defense communication).\\
\\
During the \textbf{weekly technical meeting}, we discuss our
advancements in the respective project and then we split by
teams for greater efficiency.

\subsubsection{Communication}

Our \textbf{mailing list} \href{mailto:lavieestunjeu@googlegroups.com}{lavieestunjeu@googlegroups.com}
is our main communication tool (in French), but
Regarding the technical side of our project MUST be on
From GitHub (see below) and written in English.\\
\\
We also communicate regularly GTalk, IRC, phone
and Google Hangout.

\subsubsection{Server Development}

As some technologies that we use, particularly Ocsigen do
are not portable and are very difficult to install, we use
common development server on which are installed
software and libraries necessary for developers.

\section{Sources of the projects}

\subsection{Git}

All our projects are versioned using the release manager
\textbf{Git}.\\
\\
Many references exist on the Internet for these notions, including
\href{http://fr.wikipedia.org/wiki/Gestion_de_versions}{Wikipedia page
version management}
and \href{http://git-scm.com/}{official website Git}.

\subsection{GitHub}

The host of our Git repositories called \textbf{GitHub}. This service
offers free public repositories and private repositories paying
as well as many other features.\\
\\
The main features that we use are:
\begin{description}
  \item [Organizations]: \\
    The ``organisms'' GitHub repositories are used to group
    on the same page and manage teams.\\
    Our ``organization'': \url{https://github.com/LaVieEstUnJeu}.
  \item [Issues]: \\
    The ``issues'' can be compared to ``tickets''. They allow
    to identify and discuss the features under development,
    bugs, ask questions, suggest new features
    or manage ``pull-request'' (addition of external requests
    changes to the main repository).\\
\end{description}

\subsection{Access to repositories}

The majority of our repositories is private. To access them, you must be a member
team development project.\\
\\
If you are not yet a member teams allowing you to have access
repositories, apply as soon as possible.\\
Meanwhile, in order to allow new developers who have not yet
access to our repositories (and lab EIP) to browse the contents of our repositories,
we create a user account \textbf{lavieestunjeu-watcher}. The
password for this account is ``2h\&1H52Z''.\\
\\
The current teams are:
\begin{description}
  \item [Watchers]: \\
    People outside the project with special permission
    for \textbf{see} some repositories (eg in the context of jury
    EIP official advisers, lab EIP).
  \item [Web Service API]: \\
    Developers of Web Service API compliance.
  \item [Mobile]: \\
    Developers of mobile applications and tablets.
  \item [Website]: \\
    Developer Web Client.
\end{description}

\subsection{Repositories Terms of Use}

\subsubsection{General Rules}

\begin{itemize}
  \item The contents of private repositories is confidential and must not
    not be released.
  \item The content of public repositories are protected by their licenses
    respective open-source, directly present in the repositories.
  \item To facilitate the integration of new members of nationalities
    different, we use the \textbf{English} on these repositories.\\
    No word French or another language is not tolerated, either
    the sources in the ``issues'' or ``commit messages.''
  \item Each repository and each sub-project in the repositories should contain
    a \textbf{README}. See section ``Documentation'' for
    more information.
\end{itemize}

\subsubsection{``Commit''}

\begin{itemize}
  \item commits responding to an ``issue'' after the name with a ``\#''
  \item commit messages should always be explicit
  \item It is recommended to segment the commit messages with a
    summary of less than 50 letters followed by two newlines
    and a complete description (usually a list)
  \item A commit must contain one and only one feature
  \item If the repositories contain several projects
    it is imperative to put a ``tag'' at the beginning of the commit message \\
    Example: ``[Portal] Add labels and sub-labels in the menu''
  \item ``diff'' commit should never contain lines
    off-topic.
\end{itemize}

\subsubsection{``Issues''}

\begin{itemize}
  \item There must always be an ``open'' outcome for each task in
  \item ``The issues contain tags'':
    \begin{itemize}
      \item task
      \item bug
      \item bonus
      \item issue
    \end{itemize}
    \item We do not use issues in public repositories
\end{itemize}


%% ------------------------------------------------ ---------------------%%

\chapter{Developer API Public}

\section{API}

The interraction between the data and we handle our services is
strictly defined by an API.

\subsection{Specifications}

\subsubsection{REST Architecture}

This API uses the standardized architectural style \textbf{REST}
(``Representational State Transfer'').\\
\\
Many references exist on the Internet for these notions, including
\href{http://fr.wikipedia.org/wiki/Representational_State_Transfer}{Wikipedia page}.

\subsubsection{Formatting JSON}

Data published by the API are formatted using the format
standardized data \textbf{JSON} (JavaScript Object Notation).\\
\\
Many references exist on the Internet for these notions, including
\href{http://www.json.org/}{official website}.

\subsection{API Documentation}

Our API is constantly evolving, unlike this documentation is
why it is not directly transferred to it.\\
\\
It is available on Google Docs: \url{http://goo.gl/uxMoJ}.\\
When our service is published, the API will be available on our website
Web

\section{The Web Service}

We offer our users a web service API respecting their
to manipulate data services. Our customers and web
mobile are also based on Web Service for handling
data.\\
\\
For more information on what a Web service, many
references exist on the Internet for these notions, including
\href{http://fr.wikipedia.org/wiki/Service_Web}{Wikipedia page}.

\subsection{Sources}

\subsubsection{Programming Languages}

Our Web Service is implemented in \textbf{Javascript}
the framework \textbf{Node.js}.\\
\\
The choice of this technology is detailed in the specifications.\\
\\
Many references exist on the Internet for these notions, including
\href{http://fr.wikipedia.org/wiki/JavaScript}{Javascript Wikipedia page}
and \href{http://nodejs.org/}{official website Node.js}.

\subsubsection{Repositories}

The GitHub repository of Web Service \url{https://github.com/LaVieEstUnJeu/API} \\
It is important to take familiarized the information provided in the
``README''.\\
More information about repositories in the ``Sources'' projects.

\subsection{Database}

We use a database called ``no'' relational called MongoDB. \\
\\
Many references exist on the Internet for these notions, including
\href{http://fr.wikipedia.org/wiki/NoSQL}{Wikipedia page and NOSQL}
\href{http://www.mongodb.org/}{official website MongoDB}.\\
\\
The database evolves over the progress of the project and we
API Modified regularly throughout the development phase.\\
\\
Conceptual design of the database is available in the document
architecture project: \url{https://github.com/LaVieEstUnJeu/Doc/tree/master/doc/AA} \\
\\
The current version - but not final - the database is MongoDB
available on the repository:
\url{https://github.com/LaVieEstUnJeu/API/blob/master/public/models.js}.

\section{Using the Web Service}

Our Web Service is used by our customers and web and mobile
potentially by other developers in the future. For this reason,
we decided to publish the source code using the API
then subsequently propose other examples so that users
be guided at best.\\
\\
The repository \textbf{public using the API}:
\url{https://github.com/LaVieEstUnJeu/Public-API/} \\
It is important to take familiarized the information provided in the
``README'' and ``all other README'' available examples for each entry
provided.\\
More information about repositories in the ``Sources'' projects.\\
\\
The source code for these examples is available ONLY on the repository to
not to dupplication. For this reason, using the repositories
code should be calling upon installation.

%% ------------------------------------------------ ---------------------%%

\chapter{Tablets and Mobile Applications}

\section{Sockets and technology}

Our apps are available on mobiles and tablets, several
Operating Systems\\
\\
To minimize the amount of code generated and thus limit the
amount of potential bugs, we use technology \textbf{Mono}
C\#.\\
In particular, we use:
\begin{itemize}
  \item Native code C\# to \textbf{Windows Phone 7}
  \item Mono.android for \textbf{Android}
  \item MonoTouch for \textbf{iOs}
\end{itemize}

The choice of this technology is detailed in the specifications.\\
\\
Many references exist on the Internet for these notions, including
\href{http://msdn.microsoft.com/en-us/vstudio/hh341490.aspx}{documentation
Official C\#} and \href{http://www.mono-project.com/}{official website Mono}.

\section{Source}

\subsection{Repositories}

The GitHub repository applications:
\url{https://github.com/LaVieEstUnJeu/Applications} \\
More information about repositories in the ``Sources'' projects.

\subsection{Unit Tests API}

Applications filing also proposes to implement unit tests
client-side API.

\section{Design}

The design of the views and the description of the features thereof
defined in this document are CONFIDENTIAL:
\url{http://goo.gl/oY9se}

%% ------------------------------------------------ ---------------------%%

\chapter{Website}

\section{Technology}

The website is developed with \textbf{Ocsigen}, server and web framework
in \textbf{OCaml}.\\
\\
The choice of this technology is detailed in the specifications.\\
\\
Many references exist on the Internet for these notions, including
\href{http://ocaml.org/}{official website OCaml} and
\href{http://ocsigen.org/}{official website Ocsigen}.\\
\\
Client code and server code are developed
with the same technology. The views are responsives, ie
the display is automatically adapted to display on
said ``desktop environment'', ``mobile'' or ``tablets''.

\section{Repositories}

The GitHub repository website:
\url{https://github.com/LaVieEstUnJeu/Website} \\
It is important to take familiarized the information provided in the
``README''.\\
More information about repositories in the ``Sources'' projects.

\section{URLs}

URLs are defined on the ``README'':
\url{https://github.com/LaVieEstUnJeu/Website}

\section{Views Design}

The design of the views defined in this album CONFIDENTIAL:
\href{http://photos.db0.fr/?album=GLifeDesign?authkey=Gv1sRgCLmfvp7DmMK_hAE}{album}.

\section{Features}

The description of the features on the views defined in the ``specifications''.

%% ------------------------------------------------ ---------------------%%

\chapter{Documentation}

This chapter is intended for technical writers project.\\
In the section ``code'' is open to all developers.

\section{Overall project documentation}

\subsection{Sources}

The GitHub repository overall project documentation:
\url{https://github.com/LaVieEstUnJeu/Applications} \\
It is important to take familiarized the information provided in the
``README''.\\
More information about repositories in the ``Sources'' projects.

\subsection{Steps and technologies}

We document our projects according to the wishes of the lab and add EIP
also other documentation we deem useful.

\subsubsection{Writing}

If necessary, we write the content of our documents using a
collaborative tools (Google Docs / Drive) allowing us to
work and more distance.\\
\\
If only one person working on the document, it is not
necessary to use this tool and we directly use the
layout tools.

\subsubsection{Layout}

We set our page documents using \textbf{LaTeX}
a markup language.\\
\\
Many references exist on the Internet for these notions, including
\href{http://fr.wikipedia.org/wiki/Langage_de_balisage}{the Wikipedia page} and
\href{http://www.latex-project.org/}{official website LaTeX}.

\section{Documentation directly related to repositories}

Each repository and each sub-project in the repositories should contain
documentation in README ``file''.

\subsection{Content}

This file should contain:
\begin{itemize}
  \item content, the purpose of repository / project
  \item The project dependencies
  \item A simple and quick to install, compile, run, test
  \item The tree and the location of the main
  \item An explanation for people who join the project
  \item ``coding-style'' (or coding convention) Project
  \item A FAQ
  \item The copyright (for repositories open-source)
\end{itemize}

\subsection{Sources}

These documents must be written using the language
markup \textbf{Markdown} (recommended) or another
language supported by GitHub.\\
\\
The list of languages supported by GitHub and links to
their documentation are available on
\href{https://github.com/github/markup#markups}{GitHub documentation}.

\section{In the code}

It is important to comment your code and documenting best.
To do this, keep in mind that other people will need
ironing behind to read, understand and modify, and they
must do so without difficulty and without the help of developers
Original\\
\\
We generally recommend using the syntax \textbf{Doxygen}
to automatically generate documentation, but each project
has its own coding conventions defined in the ``README''.\\
\\
Many references exist on the Internet for these notions, including
\href{http://fr.wikipedia.org/wiki/R%C3%A8gles_de_codage}{Wikipedia page
coding conventions} and \href{http://www.doxygen.org/}{official website Doxygen}.

%% ------------------------------------------------ ---------------------%%

\chapter{Go further}

This document is a simple guide for new developers
join the project without difficulty, but it is not him
only complete documentation.\\
\\
It is therefore essential that as developers, you take the time
to read about the projects you are working on using
the links in this document and doing your own research.\\
\\
It is also important for the cohesion of the group to communicate
within teams. For this, participation in meetings is mandatory
and can ask questions or answer.\\
\\
Finally, it is important that you take the time, as did
and make the team, write documentation for your projects
respective and holding them up to date.

\end{document}
