\documentclass{life-en}

\begin{document}

\title{Life is a game}
\subtitle{Analysis of the existing}
\member{Lepage Barbara}{db0company@gmail.com}
\member{Caradec Guillaume}{guillaume.caradec@gmail.com }
\member{El-Outmani Youssef}{youssef.eloutmani@gmail.com}
\member{Glorieux François}{fra.glorieux@gmail.com}
\member{Klarman Nicolas}{nickoas@gmail.com}
\member{Lassagne David}{david.lassagne@gmail.com}
\member{Le-Cor Wilfried}{wilfried.lecor@gmail.com}
\member{Lenormand Frank}{lenormf@gmail.com}
\member{Louvigny Guillaume}{guillaume@louvigny.fr}

\summary
{
  This document is the analysis of the existing projects related to our
  project. It will help us analysing the buisness plan.
}

\maketitle

\tableofcontents

\chapter{EIP Reminder}

\section{What is an “EIP” ?}

Epitech is providing an original formation to become an IT Expert. Always working on real projects, students learn to react and adapt their reactions depending of the situation for their future projects.\\

For example, they will not be lost with technologies and languages they don’t know, they will just use the usual process they learn at school.\\
\\

EIP means “Epitech Innovative Projects”. It is the most important step of Epitech studies. It started at the beginning of the third year and ended the last year. This project is achieved by a group of 6 students minimum.\\

They work together for one goal : being innovative.\\
\\

Moreover, the EIP show students the world of buisness.

\section{Our EIP : “Life is a Game”}

As part of our EIP we chose to create an entertaining social network based on \textbf{bucket lists}. A bucket list is a list of "things you want to do before dying" : it define every things his author want to do of his life, a memo to not waste his life.\\

Our project will let our visitors to make their own lists, validates their exploits, and let them publish these on other social networks. Thus, every action made by a user (adding an activity, success, failure) will be a conversation thread where his network and him would be able to discuss and share medias (photos, videos and so on). An user's activity will be validated by his own network, and will be shown as an achievement, like in video games.\\

In the future the website will expand with other functionalities only existing in video games.

\chapter{Existing projects}

\section{List of projects}

\begin{itemize}

  \section{EpicWin}

  \item \url{http://www.rexbox.co.uk/epicwin/}\\
    \begin{itemize}
      \item\textbf{EpicWin} is an application available only on iOs.
      \item It is designed to look like a Viking world.
      \item By the fact that it is an iOs application, it is not really open. There is no API to get informations about users and their interests and nobody can work on it and provide new features.
      \item You can only suscribe to everydays activities like doing the dishes, practice sport, \dots Users can’t be proud of their special achievements and can’t share with their friends what they really want to share with their friends!
      \item The last update of the project was more than one year ago so we think the project has been given up.
      \item The drawback of this project is that it is not available on Android, Windows and Blackberry phone. It is not even available on a simple internet browser.
    \end{itemize}

    \section{Bucketlist.org}

    \item \url{http://bucketlist.org/}\\
      \begin{itemize}
        \item \textbf{Bucketlist.org} is a website in which you can write your bucket list.
        \item You can say if you already have done it or if you are going to do it in the future.
        \item Achievements are provided by users. Everybody can be part of the achievement.
        \item It is not really a social network and people can’t chat or share photos.
        \item The only thing relative to a social network is the “like” button.
      \end{itemize}

      \section{Bucketlist.net}

    \item \url{http://bucketlist.net/}\\
      \begin{itemize}
        \item \textbf{Bucketlist.net} is a sharing website, not a social network.
        \item It allow you to write your bucket list.
        \item People can help you with your goals and you can help them too.
        \item It is not a game, it is not funny at all.
        \item Also, sharing experiences is not really easy with this website.
      \end{itemize}

      \section{OneFeat}

    \item \url{http://onefeat.com/}\\
      \begin{itemize}
        \item \textbf{OneFeat} is a website where you can upload funny photos of particular events.
        \item It is really funny but it is not possible to plan things by writing a bucket list or something else.
        \item It is a game because you have to share photos difficult to have and you win points.
        \item The photos is more important than the fact that you have success in you goals, and we don’t think it is a good idea to restrict a fact by a photo.
        \item It is not social at all, you play alone by yourself.
      \end{itemize}

\end{itemize}


\chapter{Our Project}

\section{What will be in our project}

We have evaluate existing projects and none of them have all the elements we want to put in our project.\\

We want our project to be a so funny game that people will come everyday.\\

We want people to share and chat, share photos and quotes.\\

We want it to be social.\\

We want our project to be intelligent and linked with the other social network to find informations about users and provide us achievements to add in their missions that correspond to their profile.\\

Achievements are called missions and missions are grouped in adventures.

\section{What will not be in our project}

We are a social network, so even if it’s a game, we  will not implement all particularity of a game.\\

For example, we will not make flash animation, shoot’em’up or strategy game.\\

People will not help the other to achieve their goals in our project.

\chapter{Conclusion}

To conclude, our project is a mix of many available projects in one application.\\


It is using what people like nowadays : social networking.\\


Also, it is giving people a chance to transform their boring daily life in an awesome everydays adventures!

\end{document}
